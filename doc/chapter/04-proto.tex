\makeatletter
	\DeclareDocumentCommand{\desktopmockup}{smmO{\thesubsectionnamely}s}{
		\begin{tikzsetheight}[\dimexpr\measurepage-1cm\relax]%
			\node[nosep, outer sep=1mm, anchor=north west] at(0.05,0.95) (upleft)
				{\includegraphics[width=0.8\textwidth, keepaspectratio, max height=0.4\measurepage]{src/imgs/mockups/#2}};

			\node[nosep, outer sep=1mm, anchor=south east] at(0.95,0.05) (downright)
				{\includegraphics[width=0.8\textwidth, keepaspectratio, max height=0.4\measurepage]{src/imgs/mockups/#3}};

			\begin{pgfonlayer}{background}
				\path[opacity=0.65, draw=ddchaptercolor!60!black, line width=2pt]
					(upleft.north west) rectangle (upleft.south east);

				\IfBooleanF{#1}{\path[fill=ddchaptercolor!45!white]%
					(0.15,0.498) rectangle (0.85,0.502); }

				\path[opacity=0.65, draw=ddchaptercolor!60!black, line width=2pt]
					(downright.north west) rectangle (downright.south east);
			\end{pgfonlayer}
		\end{tikzsetheight}%
		\captionof{figure}{\begingroup\color{ddchaptercolor!80!black}\texttt{#4}\endgroup}%
		\IfBooleanF{#5}{\clearpage\mbox{}}
	}

	\DeclareDocumentCommand{\mobilemockup}{smmO{\thesubsectionnamely}s}{
		\begin{tikzsetheight}[\dimexpr\measurepage-1cm\relax]
			\node[nosep, outer sep=1mm, anchor=north west] at(0.05,0.95) (upleft)
				{\includegraphics[height=0.70\measurepage, max width=0.35\pagewidth, keepaspectratio]{src/imgs/mockups/#2}};
			\node[nosep, outer sep=1mm, anchor=south east] at(0.95,0.05) (downright)
				{\includegraphics[height=0.70\measurepage, max width=0.35\pagewidth, keepaspectratio]{src/imgs/mockups/#3}};

			\begin{pgfonlayer}{background}
				\path[opacity=0.65, line width=2pt, draw=ddchaptercolor!60!black]
					(upleft.north west) rectangle (upleft.south east);

				\IfBooleanF{#1}{\path[fill=ddchaptercolor!45!white]%
					(0.498,0.15) rectangle (0.502,0.85); }

				\path[opacity=0.65, line width=2pt, draw=ddchaptercolor!60!black]
					(downright.north west) rectangle (downright.south east);
			\end{pgfonlayer}

		\end{tikzsetheight}
		\captionof{figure}{\begingroup\color{ddchaptercolor!80!black}\texttt{#4}\endgroup}
		\IfBooleanF{#5}{\clearpage\mbox{}}
	}

	\DeclareDocumentCommand{\mobilemockupelongated}{smmO{\thesubsectionnamely}}{
		\begin{tikzsetheight}[\dimexpr\measurepage-1cm\relax]
			\node[nosep, outer sep=1mm, anchor=north west] at(0.15,0.95) (upleft)
				{\includegraphics[height=0.825\measurepage, max width=0.45\pagewidth, keepaspectratio]{src/imgs/mockups/#2}};
			\node[nosep, outer sep=1mm, anchor=south east] at(0.85,0.05) (downright)
				{\includegraphics[height=0.825\measurepage, max width=0.45\pagewidth, keepaspectratio]{src/imgs/mockups/#3}};

			\begin{pgfonlayer}{background}
				\path[opacity=0.65, line width=2pt, draw=ddchaptercolor!60!black]
					(upleft.north west) rectangle (upleft.south east);

				\IfBooleanF{#1}{\path[fill=ddchaptercolor!45!white]%
					(0.498,0.15) rectangle (0.502,0.85); }

				\path[opacity=0.65, line width=2pt, draw=ddchaptercolor!60!black]
					(downright.north west) rectangle (downright.south east);
			\end{pgfonlayer}

		\end{tikzsetheight}
		\captionof{figure}{\begingroup\color{ddchaptercolor!80!black}\texttt{#4}\endgroup}
		\clearpage\mbox{}%
	}

\makeatother

\chapter{Mock-up}(midpoint-4)
<Di seguito riportati una serie di mock-up dell'interfaccia front-end prevista; in quanto si prevede
che l'applicazione debba poter essere utilizzata sia su dispositivi mobile che widescreen, verranno prodotti
mock-up che ne visualizzano i viewport\footnote{Viewport: porzione di una schermata \textit{a vista} per l'utente; ci si sta riferendo alla quantità di contenuto visibile
per una determinata interfaccia.} per entrambe le versioni.>

\section*{Criteri di design}+\Materialforum+
\subsection*{Adaptive design}+\Materialrotateright+[Viewport(s)]
Per motivi di utilizzabilità dell'applicazione, in quanto non vi è un dispositivo-target
specifico per essa, verranno presentati mockup che rappresentano, in modo più o meno significativo,
la disposizione delle componenti per i due diversi viewport \textit{estremi}\footnote{i dispositivi desktop, 
non-mobile, presentano schermi di grosse dimensioni su cui possono figurare più elementi; i dispositivi
mobile si presentano con le caratteristiche opposte, per l'appunto \textit{estremi}.} di utilizzo.
\begin{description}
	\item[Widescreen] Per dispositivi i cui display presentano superfici di discrete dimensioni
		che non presentano quindi limiti.
	\begingroup\color{ddchaptercolor}\vspace{6pt}\hrule height 0.6pt depth 0pt width \textwidth\relax\vspace{6pt}\endgroup
	\item[Mobile] Per dispositivi con schermi di dimensioni ridotte, che quindi concentrano
		una miriade di componenti su di una superfice piccola.
		\par\vspace{3mm}
		Stiamo ovviamente parlando di dispositivi tascabili hand-held\footnote{Hand-held: \textit{tenuto in mano},
		ad esempio palmari, tablet di piccole dimensioni.} e dispositivi cellulari d'uso comune.
\end{description}

\subsection*{Selezione dei colori}+\Materialcolorlens+[Palette]
La scelta della palette\footnote{Palette: termine di orgine francese, la cui traduzione è \textit{tavolozza}, prende
il nome dall'omonimo strumento artistico; presenta la gamma di colori utilizzata per il design e lo styling dell'applicazione.} 
per quanto concerne l'applicazione è avvenuta con i seguenti criteri
\begin{enumerate}
	\item Il testo deve avere un buon contrasto, garantendone la leggibilità in qualsiasi condizione
		di luminosità ambientale.
	\item In condizioni di scarsa luminosità, ad esempio durante un periodo d'attività notturno
		del locale, l'interfaccia non deve risultare troppo \textit{accesa} tanto da infastidirne
		l'operatore-dipendente.
\end{enumerate}

\begin{tikzsetheight}{white}
	\coordinate (cc) at (0.5, 0.5);
	\def\phaseatangle{18}
	\def\phase{6mm}
	\coordinate (lightcc) at ([shift={(\phaseatangle:\phase)}] cc);
	\coordinate (darkcc) at ([shift={(\phaseatangle:-\phase)}] cc);
	\def\inner{4.5mm}
	\def\outer{32mm}
	\def\lightshades{e5e5e5, 656565, 040404, 39795d, 395d5d}
	\def\darkshades{1c3132, 263b3b, 39795D, 39885d, e5e5e5}

	\pgfmathsetmacro{\totalradius}{115}
	\pgfmathsetmacro{\labelatangle}{330}
	\foreach \i [count=\n from 1] in \lightshades {\global\let\nshades=\n}
	\pgfmathsetmacro{\amountpercolor}{\totalradius/\nshades}

	% Light shades
	% \path[fill=white, draw=green, line width=2pt] (lightcc) circle(\outer);
	\path[fill=white, draw=green, line width=1pt] (lightcc) circle(\inner);
	\foreach \shade[count=\ix from 0, 
		evaluate=\ix as \sangle using {\amountpercolor*\ix}] in \lightshades{
			\definecolor{tmpshade}{HTML}{\shade}
			\path[draw=green, fill=tmpshade, line width=1pt] ([shift={(\sangle:\inner)}] lightcc)
				-- ([shift={(\sangle:\outer)}] lightcc)
				arc[start angle=\sangle, delta angle=\amountpercolor, radius=\outer]
				-- ([shift={(\sangle+\amountpercolor:\inner)}] lightcc)
				arc[start angle=\sangle+\amountpercolor, delta angle=-\amountpercolor, radius=\inner]
				-- cycle;
	}

	% Label
	\node[nosep, anchor=north west] (lightcclbl) at ([shift={(\labelatangle:20mm)}] lightcc)
		{\Lato\fontsize{8}{0}\selectfont Light palette};
	\draw[-stealth, line width=0.5pt, green] ([xshift=3pt, yshift=-2pt] lightcclbl.south east)
		-- ([xshift=-2pt, yshift=-2pt] lightcclbl.south west)
		-- ([xshift=-2pt] lightcclbl.north west)
		-- ([shift={(\labelatangle:\inner+2pt)}] lightcc);

	% Dark shades
	% \path[fill=white, draw=green, line width=2pt] (darkcc) circle(\outer);
	\path[fill=black!90!white, draw=black, line width=1pt] (darkcc) circle(\inner);
	\foreach \shade[count=\ix from 0, 
		evaluate=\ix as \sangle using {\amountpercolor*\ix+180}] in \darkshades{
			\definecolor{tmpshade}{HTML}{\shade}
			\path[draw=black, fill=tmpshade, line width=1pt] ([shift={(\sangle:\inner)}] darkcc)
				-- ([shift={(\sangle:\outer)}] darkcc)
				arc[start angle=\sangle, delta angle=\amountpercolor, radius=\outer]
				-- ([shift={(\sangle+\amountpercolor:\inner)}] darkcc)
				arc[start angle=\sangle+\amountpercolor, delta angle=-\amountpercolor, radius=\inner]
				-- cycle;
	}

	\node[nosep, anchor=south east] (darkcclbl) at ([shift={(\labelatangle:-20mm)}] darkcc)
		{\Lato\fontsize{8}{0}\selectfont Dark palette};
	\draw[-stealth, line width=0.5pt, green] ([xshift=-3pt, yshift=2pt] darkcclbl.north west)
		-- ([xshift=2pt, yshift=2pt] darkcclbl.north east)
		-- ([xshift=2pt] darkcclbl.south east)
		-- ([shift={(\labelatangle:-\inner-2pt)}] darkcc);
\end{tikzsetheight}

\section*{Widescreen Viewport}+\Materialpanoramahorizontal+<Widescreen>

\subsection{Autenticazione}[Inerente al requisito funzionale \ref{reqsf:authlogin}]
\desktopmockup*{Desktop/Desktop_login_dark}{Desktop/Desktop_login_light}

\subsection{Pannello amministrativo}
% \desktopmockup{Desktop/Admin_panel_empty_closed_drawer_dark}{Desktop/Admin_panel_empty_closed_drawer_light}
% 	[Pannello amministrativo (home, pannello laterale chiuso)]

\desktopmockup{Desktop/Admin_panel_empty_opened_drawer_dark}{Desktop/Admin_panel_empty_opened_drawer_light}
	[Pannello amministrativo (home, pannello laterale aperto)]

\subsection{Registrazione, disattivazione dei dipendenti}[Inerente ai requisiti funzionale \ref{reqsf:register}, \ref{reqsf:unregister}]
\desktopmockup{Desktop/Admin_panel_registration_no_data_dark}{Desktop/Admin_panel_registration_no_data_light}
	[Registrazione dipendente]

\desktopmockup{Desktop/Admin_panel_registration_dark}{Desktop/Admin_panel_registration_light}
	[Registrazione dipendente con esempio di completamento del form]

\desktopmockup{Desktop/Admin_panel_account_management_dark}{Desktop/Admin_panel_account_management_light}
	[Gestione account dipendenti]

\subsection{Visualizzazione del rendimento}[Inerente al requisito funzionale \ref{reqsf:stats}]
\desktopmockup{Desktop/Admin_panel_statistics_no_data_dark}{Desktop/Admin_panel_statistics_no_data_light}
	[Visualizzazione rendimento (nessun profilo selezionato)]

\desktopmockup{Desktop/Admin_panel_statistics_dark}{Desktop/Admin_panel_statistics_light}
	[Visualizzazione rendimento (comparazione tra due chef fittizi)]

\subsection{Visualizzazione, accettazione e notifica completamento ordini}[Inerente al requisito funzionale \ref{reqsf:acptorder}, \ref{reqsf:cmporder}]
\desktopmockup{Desktop/Cook_panel_main_menu_dark}{Desktop/Cook_panel_main_menu_light}
	[Pannello del cuoco (home)]

\desktopmockup{Desktop/Cook_panel_main_menu_filled_waiting_orders_dark}{Desktop/Cook_panel_main_menu_filled_waiting_orders_light}
	[Visualizzazione ed accettazione degli ordini in attesa]

\desktopmockup{Desktop/Cook_panel_main_menu_filled_accepted_orders_dark}{Desktop/Cook_panel_main_menu_filled_accepted_orders_light}
	[Visualizzazione e possibilità di notifica completamento degli ordini accettati]

\desktopmockup{Desktop/Cook_panel_filled_completed_orders_dark}{Desktop/Cook_panel_filled_completed_orders_light}
	[Visualizzazione ordini evasi]

\subsection{Creazione degli ordini}[Inerente al requisito funzionale \ref{reqsf:addorder}]
\desktopmockup{Desktop/Waiter_panel_table_dashboard_dark}{Desktop/Waiter_panel_table_dashboard_light}
	[Dashboard del cameriere (con selezione dei tavoli)]

\desktopmockup{Desktop/Waiter_panel_table_dashboard_menu_open_dark}{Desktop/Waiter_panel_table_dashboard_menu_open_light}
	[Dashboard del cameriere (con menù a tendina aperto)]

\desktopmockup{Desktop/Waiter_panel_menu_dark}{Desktop/Waiter_panel_menu_light}
	[Menù degli alimenti (creazione di un ordine)]

\desktopmockup{Desktop/Waiter_panel_menu_open_item_dark}{Desktop/Waiter_panel_menu_open_item_light}
	[Inserimento di un alimento per il tavolo]

\desktopmockup{Desktop/Waiter_panel_menu_order_sent_dark}{Desktop/Waiter_panel_menu_order_sent_light}
	[Conferma di invio avvenuto dell'ordine]

\subsection{Aggiornamento della cucina}[\ref{reqsf:addmenu}, \ref{reqsf:delmenu}, \ref{reqsf:chgmenu}, \ref{reqsf:addtbl}, \ref{reqsf:deltbl}, \dots]
\desktopmockup{Desktop/Admin_panel_dish_and_tables_1tab_dark}{Desktop/Admin_panel_dish_and_tables_1tab_light}
	[Inserimento tavoli, categorie e inserimento manuale pietanze]

\desktopmockup{Desktop/Admin_panel_dish_and_tables_2tab_dark}{Desktop/Admin_panel_dish_and_tables_2tab_light}
	[Aggiorna tavoli, categorie e inserimento automatizzato delle pietanze]

\desktopmockup{Desktop/Admin_panel_dish_and_tables_2tab_OFF_dark}{Desktop/Admin_panel_dish_and_tables_2tab_OFF_light}
	[Esempio di inserimento di pietanza tramite inserimento automatizzato]

\desktopmockup{Desktop/Admin_panel_dish_and_tables_3tab_no_data_dark}{Desktop/Admin_panel_dish_and_tables_3tab_no_data_light}
	[Aggiornamento delle categorie, delle pietanze]

\desktopmockup{Desktop/Admin_panel_dish_and_tables_3tab_filled_dark}{Desktop/Admin_panel_dish_and_tables_3tab_filled_light}
	[Aggiornamento delle categorie, delle pietanze (con esempio)]

\desktopmockup{Desktop/Admin_panel_dish_and_tables_4tab_no_data_dark}{Desktop/Admin_panel_dish_and_tables_4tab_no_data_light}
	[Cambio rapido delle categorie delle pietanze]

\desktopmockup{Desktop/Admin_panel_dish_and_tables_4tab_filled_dark}{Desktop/Admin_panel_dish_and_tables_4tab_filled_light}
	[Cambio rapido delle categorie delle pietanze (con esempio di selezione)]
%
\subsection{Aggiornamento delle credenziali d'accesso alla prima autenticazione}[Inerente al requisito funzionale \ref{reqsf:authupdate}]
\desktopmockup{Desktop/Desktop_change_password_dark}{Desktop/Desktop_change_password_light}
	[Aggiornamento delle credenziali]

\desktopmockup{Desktop/Desktop_change_password_success_dark}{Desktop/Desktop_change_password_success_light}
	[Aggiornamento delle credenziali con notifica]

\subsection{Inserimento, visualizzazione e nascondimento avvisi}[Inerente ai requisiti funzionali \ref{reqsf:addntf},\ref{reqsf:hidentf}]
\desktopmockup{Desktop/Admin_panel_notifications_dark}{Desktop/Admin_panel_notifications_light}
	[Inserimento avvisi (accessibile dai soli admin, vuoto)]

\desktopmockup{Desktop/Employee_menu_notifications_empty_dark}{Desktop/Employee_menu_notifications_empty_light}
	[Visualizzazione avvisi (vuoto)]

\desktopmockup{Desktop/Employee_menu_notifications_filled_dark}{Desktop/Employee_menu_notifications_filled_light}
	[Visualizzazione avvisi (ad espansione avvenuta)]*

\section*{Mobile Viewport}+\Materialpanoramavertical+<Mobile>
\subsection{Autenticazione}[Inerente al requisito funzionale \ref{reqsf:authlogin}]
\mobilemockup*{Mobile/Mobile_login_dark}{Mobile/Mobile_login_light}

% \subsection{Pannello amministrativo}
% \mobilemockup{Mobile/Mobile_admin_home_dark}{Mobile/Mobile_admin_home_light}
% 	[Pannello amministrativo]

\subsection{Registrazione, disattivazione dei dipendenti}[Inerente ai requisiti funzionale \ref{reqsf:register}, \ref{reqsf:unregister}]
\mobilemockup{Mobile/Mobile_admin_registration_dark}{Mobile/Mobile_admin_registration_light}
	[Registrazione dipendente]

\mobilemockup{Mobile/Mobile_admin_edit_account_dark}{Mobile/Mobile_admin_edit_account_light}
	[Gestione account dipendenti]

\subsection{Visualizzazione del rendimento}[Inerente al requisito funzionale \ref{reqsf:stats}]
\mobilemockup{Mobile/Mobile_statistics_empty_dark}{Mobile/Mobile_statistics_empty_light}
	[Visualizzazione rendimento (nessun profilo selezionato)]

\mobilemockup{Mobile/Mobile_statistics_no_data_dark}{Mobile/Mobile_statistics_no_data_light}
	[Visualizzazione rendimento (nessun profilo trovato per l'utente selezionato)]

\mobilemockupelongated{Mobile/Mobile_statistics_filled_dark}{Mobile/Mobile_statistics_filled_light}
	[Visualizzazione rendimento (comparazione tra due chef fittizi)]

\subsection{Visualizzazione, accettazione e notifica completamento ordini}[Inerente al requisito funzionale \ref{reqsf:acptorder}, \ref{reqsf:cmporder}]
\mobilemockup{Mobile/Mobile_cook_main_dark}{Mobile/Mobile_cook_main_light}
	[Pannello del cuoco (home)]

\subsection{Creazione degli ordini}[Inerente al requisito funzionale \ref{reqsf:addorder}]
\mobilemockup{Mobile/Mobile_waiter_dashboard_dark}{Mobile/Mobile_waiter_dashboard_light}
	[Dashboard del cameriere (con selezione dei tavoli)]

\mobilemockup{Mobile/main_waiter_order_component_dark}{Mobile/main_waiter_order_component_light}
	[Menù degli alimenti (per la creazione di un ordine)]

\mobilemockup{Mobile/main_waiter_order_component_expanded_dark}{Mobile/main_waiter_order_component_expanded_light}
	[Menù degli alimenti (per la creazione di un ordine) esteso]

\mobilemockup{Mobile/order_component_empty_cart_dark}{Mobile/order_component_empty_cart_light}
	[Dialog Carrello per il tavolo selezionato (vuoto)]

\mobilemockup{Mobile/order_component_cart_dark}{Mobile/order_component_cart_light}
	[Dialog Carrello per il tavolo selezionato (con pietanze)]

\mobilemockup{Mobile/order_component_dish_dialog_dark}{Mobile/order_component_dish_dialog_light}
	[Dialog per l'inserimento di una pietanza nel carrello]

\subsection{Aggiornamento della cucina}[\ref{reqsf:addmenu}, \ref{reqsf:delmenu}, \ref{reqsf:chgmenu}, \ref{reqsf:addtbl}, \ref{reqsf:deltbl}, \dots]
\mobilemockupelongated{Mobile/Mobile_admin_tables_dishes_1_dark}{Mobile/Mobile_admin_tables_dishes_1_light}
	[Inserimento tavoli, categorie e inserimento manuale pietanze]

\mobilemockupelongated{Mobile/Mobile_admin_tables_dishes_2_dark}{Mobile/Mobile_admin_tables_dishes_2_light}
	[Aggiorna tavoli, categorie e inserimento automatizzato delle pietanze]

\mobilemockup{Mobile/Mobile_admin_tables_dishes_3_dark}{Mobile/Mobile_admin_tables_dishes_3_light}
	[Aggiorna pietanze (nessuna pietanza selezionata)]

\mobilemockupelongated{Mobile/Mobile_admin_tables_dishes_3_filled_dark}{Mobile/Mobile_admin_tables_dishes_3_filled_light}
	[Esempio di modifica di una pietanza]

\mobilemockup{Mobile/Mobile_admin_tables_dishes_4_dark}{Mobile/Mobile_admin_tables_dishes_4_light}
	[Cambio rapido delle categorie delle pietanze]

\subsection{Visualizzazione, accettazione e notifica completamento ordini}[Inerente al requisito funzionale \ref{reqsf:acptorder}, \ref{reqsf:cmporder}]

\mobilemockup{Mobile/Mobile_cook_panel_filled_pending_orders_dark}{Mobile/Mobile_cook_panel_filled_pending_orders_light}
	[Visualizzazione ed accettazione degli ordini in attesa]

\mobilemockup{Mobile/Mobile_cook_panel_filled_pending_orders_success_dark}{Mobile/Mobile_cook_panel_filled_pending_orders_success_light}
	[Visualizzazione ed accettazione degli ordini in attesa (con notifica di avvenuta accettazione)]

\mobilemockup{Mobile/Mobile_cook_panel_filled_accepted_orders_dark}{Mobile/Mobile_cook_panel_filled_accepted_orders_light}
	[Visualizzazione e possibilità di notifica completamento degli ordini accettati]

\mobilemockup{Mobile/Mobile_cook_panel_filled_completed_orders_dark}{Mobile/Mobile_cook_panel_filled_completed_orders_light}
	[Visualizzazione ordini evasi]

\subsection{Aggiornamento delle credenziali d'accesso alla prima autenticazione}[Inerente al requisito funzionale \ref{reqsf:authupdate}]
\mobilemockup{Mobile/Mobile_change_password_dark}{Mobile/Mobile_change_password_light}
	[Aggiornamento delle credenziali]

\mobilemockup{Mobile/Mobile_change_password_success_dark}{Mobile/Mobile_change_password_success_light}
	[Aggiornamento delle credenziali con notifica]

\subsection{Inserimento, visualizzazione e nascondimento avvisi}[Inerente ai requisiti funzionali \ref{reqsf:addntf},\ref{reqsf:hidentf}]
\mobilemockup{Mobile/Mobile_admin_notifications_dark}{Mobile/Mobile_admin_notifications_light}
	[Inserimento avvisi (accessibile ai soli admin, vuoto)]

\mobilemockup{Mobile/Mobile_closed_notification_dark}{Mobile/Mobile_closed_notification_light}
	[Visualizzazione avvisi (vuoto)]

\mobilemockup{Mobile/Mobile_expanded_notification_dark}{Mobile/Mobile_expanded_notification_light}
	[Visualizzazione avvisi (ad espansione avvenuta)]*\mbox{}
