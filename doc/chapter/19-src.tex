\chapter[Codice]{sorgente}(midpoint-7)
\section*{Briefing}+\Materialfilecopy+
\subsection*{Versioning}+\IcoMoongithub+[GitHub]
Il codice sviluppato, sia per quanto riguarda \textbf{frontend}, \textbf{backend}
ma anche della presente \textbf{documentazione} \LaTeX, sin da subito, è stato soggetto di versioning.

Il versioning è stato effettuato tramite lo strumento \texttt{git}, utilizzando 
una repository su di una macchina privata, infine il codice sorgente è stato clonato
sull'omonimo host \textbf{Microsoft \IcoMoongithub\ GitHub}.

Pertanto, il codice sorgente sviluppato, aggiornato all'ultima versione, sarà disponibile all'indirizzo\newline
\url{https://github.com/Zen920/INGSW2022-2023/}

\subsection*{Licensing}+\Materialfilecopy+[MIT License]
Il progetto è da considerarsi open-source con la licenza \textbf{MIT}: 
si è totalmente liberi di prelevare il codice sorgente, modificarlo a proprio piacimento
e di ridistribuirlo, a patto che la licenza stessa \textbf{NON} venga
modificata e che gli autori (n.d.r. I sottoscritti Amato Ciro e Lombardi Vincenzo)
vengano citati negli altrui progetti.

\begin{minted}[linenos=false, frame=none, xleftmargin=0pt]{text}
The MIT License (MIT)
Copyright © 2022-2023 Ciro Amato, Vincenzo Lombardi
Permission is hereby granted, free of charge, to any person obtaining a copy of
this software and associated documentation files (the “Software”), to deal in
the Software without restriction, including without limitation the rights to
use, copy, modify, merge, publish, distribute, sublicense, and/or sell copies
of the Software, and to permit persons to whom the Software is furnished to do
so, subject to the following conditions:
The above copyright notice and this permission notice shall be included in all
copies or substantial portions of the Software.
THE SOFTWARE IS PROVIDED “AS IS”, WITHOUT WARRANTY OF ANY KIND, EXPRESS OR
IMPLIED, INCLUDING BUT NOT LIMITED TO THE WARRANTIES OF MERCHANTABILITY,
FITNESS FOR A PARTICULAR PURPOSE AND NONINFRINGEMENT. IN NO EVENT SHALL THE
AUTHORS OR COPYRIGHT HOLDERS BE LIABLE FOR ANY CLAIM, DAMAGES OR OTHER
LIABILITY, WHETHER IN AN ACTION OF CONTRACT, TORT OR OTHERWISE, ARISING FROM,
OUT OF OR IN CONNECTION WITH THE SOFTWARE OR THE USE OR OTHER DEALINGS IN THE
SOFTWARE.
\end{minted}

\newpage
\subsection*{Installazione, esecuzione d'una istanza}+\Brandsdocker+[Docker]
Il presente progetto è stato sviluppato tenendo in considerazione la suddivisione
in micro-servizi: all'occorrenza ogni componente può essere separato ed eseguito indipendentemente
su di una macchina diversa.

Nella sottocartella \texttt{\color{seagreen}docker/}, presente alla radice della repository
indicata precedentemente, sarà possibile configurare, installare ed eseguire
rapidamente un'instanza funzionante dei vari micro-servizi necessari: c'è da precisare
che la tecnologia \Brandsdocker\ docker è dipendenza necessaria sull'host che s'intende
utilizzare per il deployment!

Sarà sufficiente lanciare il comando \texttt{\color{seagreen}docker compose up -d} dalla cartella
designata e, automaticamente,\ \Brandsdocker\ docker si preoccuperà di inizializzare tutti
i micro-servizi: potrebbero essere necessari alcuni minuti, a seconda della velocità 
della connessione, nonchè dalle prestazioni del dispositivo host utilizzato.

Infine, segue una breve descrizione di tutti i micro-servizi containerizzati attualmente in uso,
idenficati dall'hostname del contenitore.
\begin{itemize}
	\item \textbf{backend}, ossia il micro-servizio Java che esegue il backend, REST API, dell'applicazione.
	\item \textbf{frontend}, si preoccupa di compilare il codice React in codice statico, nonchè
	di configurare un'instanza nginx su cui girerà il front-end.
	\item \textbf{postgres}, è appunto il database cui è collegato il backend.
\end{itemize}
