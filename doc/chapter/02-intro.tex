\chapter{Introduzione}(midpoint-2)<La società SoftEngUniNA ha l'obbiettivo di commercializzare diversi sistemi informatici. In questo capitolo discuteremo, nello specifico, quale sistema ci è stato assegnato e quali sono state le nostre scelte implementative.>

\section{Il progetto}
Ratatouille23 è un sistema finalizzato al supporto, alla gestione di attività di ristorazione. 

L'applicazione richiesta dovrà garantire ideali di performance nonché di affidabilità i cui utenti
potranno usufruire delle funzionalità in modo intuitivo e piacevole.

\subsection{Oggetto della richiesta di Implementazione}
Nello specifico, si richiederà l'implementazione \textit{almeno} delle seguenti
funzionalità:

\begin{chapterbox}
	Le funzionalità riportate sono state assegnate personalmente al \textbf{gruppo}.
\end{chapterbox}

\begin{enumerate}
	\item Un amministratore può creare utenze per i propri dipendenti (sia addetti alla sala, che addetti
alla cucina, che supervisori). Al primo accesso, ogni utente deve re-impostare la password
inserita dall’amministratore, scegliendo una password diversa. 
	\item Un amministratore (o un supervisore) può personalizzare il menù dell’attività di ristorazione. In
particolare, l’utente può creare e/o eliminare elementi dal menù. Ciascun elemento è
caratterizzato da un nome, un costo, una descrizione, e un elenco di allergeni comuni. Inoltre,
è possibile organizzare gli elementi del menù in categorie personalizzabili (e.g.: primi, dessert,
primi di pesce, bibite, etc.), e definire l’ordine con cui gli elementi compaiono nel menù. In fase
di inserimento, è richiesto l’auto completamento di alcuni prodotti (e.g.: bibite o preconfezionati)
utilizzando open data come quelli disponibili in \href{https://it.openfoodfacts.org}{OpenFoodFacts}
	\item Un addetto alla sala può registrare ordinazioni indicando l’identificativo del tavolo e gli elementi del menù desiderati.
	\item Un addetto alla cucina può visualizzare le ordinazioni in tempo reale, procedere alla
preparazione dei piatti, e tenere traccia delle ordinazioni già evase.
	\item Un supervisore o un amministratore può inserire nel sistema degli avvisi, che vengono
visualizzati da tutti i dipendenti. Ciascun dipendente può marcare un avviso come “visualizzato” e nasconderlo.
	\item Un amministratore può visualizzare statistiche dettagliate sulla produttività del personale
addetto alla cucina. In particolare, in un lasso di tempo personalizzabile, deve essere possibile
visualizzare almeno quanti ordini ciascun addetto alla cucina ha evaso. È apprezzata la presenza
di grafici interattivi.
\end{enumerate}

\pagebreak
\subsection{Funzionalita' richieste}
Dalle funzionalità \emdashme{oggetto di richiesta} ne ricaviamo quali effettivamente saranno le richieste del progetto. In particolare si fornisce un elenco, più o meno dettagliato, di quali siano le necessità implementative riconosciute.

\begin{enumerate}
	\item Una \textit{gerarchia} di dipendenti, con diverse necessità e quindi permessi.
	\begin{enumerate}
		\item L'amministratore può creare, rimuovere dipendenti o visualizzarne il rendimento, in termini di prestazioni lavorative. Può inoltre modificare il menù.
		\item Il supervisore aggiunge avvisi relativi alla cucina e, come l'amministratore, può modificare il menù.
		\item L'addetto di sala, ha il compito di registrare nuove ordinazioni, si suppone ci possano essere più dipendenti che ricoprano questo ruolo;
		\item L'addetto alla cucina prepara le pietanze, quindi, evade i nuovi ordini; anche in questo caso si suppone esserci più dipendenti in tale ruolo.
	\end{enumerate}

	\item La registrazione del menù, quindi delle pietanze disponibili, tramite un'interfaccia dedicata che faccia uso dell'API \href{https://it.openfoodfacts.org}{OpenFoodFacts}.

	\item La creazione, rimozione e monitoraggio dei dipendenti attraverso interfacce ad hoc ad uso amministrativo.

	\item Un sistema di notifiche interno che permette ad amministratori e supervisori di comunicare in tempo reale avvisi e necessità i cui messaggi, una volta visualizzati, possano essere nascosti.

	\item La totale gestione delle ordinazioni: l'inserimento, la rimozione, quindi l'incarico e infine l'evasione dell'ordine sono compiti assai diversi; pertanto richiederanno metodologie di visualizzazione altrettanto diverse.
\end{enumerate}

\section{Proposta implementativa}
Fatte le dovute analisi, si giunge quindi alla conclusione che saranno necessari diverse componenti affinché il sistema risulti \textbf{sicuro} ed \textbf{efficiente} in un'ambiente che ne richiede il funzionamento multiutente.

\begin{chapterbox}[Disambiguazione]
	Nella sezione che segue non verranno affatto citate le tecnologie utilizzate, piuttosto il modus operandi adattato nonché la strategia e la suddivisione delle componenti.
	
	\vspace{0.25cm}
	Se si è più interessati al prodotto finale, piuttosto, passare all'argomento corrispondente.
	% TODO: Add label
\end{chapterbox}

Nello specifico, i requisiti fanno sì che la soluzione proposta debba essere esposta ad un pubblico piuttosto ridotto, in quanto i soli dipendenti abilitati potranno fare uso del sistema informativo richiesto.

Inoltre dai requisiti è necessario che tale applicazione sia di facile utilizzo e che quindi basse siano le capacità richieste nell'utilizzo, risultando adatte a novizi.

\subsection{Applicazione: Struttura dinamica}
L'applicazione proposta segue i principi di usabilità, dell'immediatezza e delle guidelines utente facendo uso di tecnologie front-end specifiche per renderla multipiattaforma.

In quanto si presuppone che l'utilizzo dell'applicazione sia riservato ai soli dipendenti dell'attuale attività di ristorazione, la welcome page \textbf{richiede} espressamente che venga effettuato il login affinché possano esserne utilizzate le funzionalità. Inoltre, data la presenza di diverse tipologie di utenti \emdashme{i così detti ruoli} sarà il login stesso a \textit{smistare} il dipendente nella sezione apposita.

Ogni dipendente avrà quindi un'interfaccia dinamica a seconda del ruolo loro assegnato.

\subsection{Applicazione: Menu amministrativo}
L'amministratore, in quanto tale, avrà un'interfaccia più o meno complicata (per numero di azioni).

Non sarà presente un'interfaccia di registrazione di dominio \textit{pubblico}, in quanto compito degli amministratori registrare nuovi dipendenti e quindi fornire i dati di accesso.

Gli amministratori sono inoltre abilitati all'inserimento, alla cancellazione e alla modifica di nuove pietanze all'interno del menù tramite interfaccia apposita.

L'amministratore può, inoltre, propagare nuovi avvisi attraverso il sistema affinché raggiunga tutto il personale impiegato.

Infine, gli amministratori possono verificare l'andamento e il rendimento dei dipendenti impiegati nell'attività ristorativa nel ruolo di \textit{addetti alla cucina}.

\pagebreak
\subsection{Applicazione: Supervisor}
Il supervisore è una figura che nella gerarchia lavorativa si colloca esattamente tra l'amministrazione e la classe dipendente \emdashme{addetta al lavoro}, figura quindi di amministrazione secondaria.

La loro interfaccia utente è riassumibile come un sottoinsieme di funzionalità dell'amministrazione, più precisamente consiste delle funzionalità di gestione del menù di sala (quindi dell'inserimento, della cancellazione e della modifica delle pietanze) e nella gestione di avvisi nel sistema.

\subsection{Applicazione: Addetto alla sala}
L'addetto alla sala è colui che si occupa di gestire le ordinazioni, nello specifico è stato interpretato come la figura che si pone tra i clienti dell'attività e gli addetti alla cucina. In ottica funzionale, il suo ruolo è ben delimitato al solo inserimento delle ordinazioni all'interno del sistema che provvederà per quanto concerne la notificazione al personale addetto alla cucina.

\subsection{Applicazione: Addetto alla cucina}
L'addetto alla cucina identifica esattamente un membro dell'attività il cui compito ultimo è la preparazione delle pietanze. In una visione applicativa, egli riceve dal sistema una o più istanze di ordine e sarà compito suo accettare, quindi preparare e notificare nuovamente il sistema che suddetto ordine sia pronto per essere servito.

\section{Ufficializzazione del progetto}
Il presente documento descrive quanto richiesto dalla società Contraente, SoftEngUniNA, ai Contraenti Amato Ciro e Vincenzo Lombardi, per la realizzazione del sistema informativo Ratatouille23.

Il committente ha richiesto una piattaforma informativa, sicura e di facile utilizzo, che permetta la gestione di attività di ristorazione. Sono previste le seguenti componenti e/o tecnologie per la sua realizzazione:
\begin{enumerate}
	\item Front-end, un applicativo multipiattaforma che permetta ai soli dipendenti dell'attività di svolgerne le attività; sono previste aree diverse, quindi interfacce dedicate, per ognuno dei ruoli.

	\item Back-end, un sistema che risiede su un elaboratore remoto, cui si interfaccia il front-end per l'inserimento o la ricezione di informazioni. Tale sistema risulta inaccessibile se non fisicamente tramite l'ausilio di strumenti terzi da parte di un amministratore. A sua volta, il back-end fa uso delle seguenti tecnologie
	\begin{enumerate}
		\item Database, che raccoglie tutte le informazioni necessarie al funzionamento del back-end.
		\item Docker, atto a semplificarne il processo di deployment e quindi la distribuzione.
	\end{enumerate}
\end{enumerate}

Le diverse aree riservate previste, caratterizzate per ruolo, sono le seguenti:
\begin{itemize}
	\item Amministrazione, nello specifico:
	\begin{enumerate}
		\item Aggiunta (registrazione) e Rimozione (disattivazione) dei dipendenti.
		\item Aggiunta e rimozione di alimenti nel menù culinario.
		\item Notificare (aggiunta di avvisi) l'intero personale-dipendente dell'attività.
		\item Visualizzare statistiche dettagliate sulle prestazioni e l'andamento degli addetti alla cucina.
	\end{enumerate}

	\item Inserimento di nuove ordinazioni all'interno del sistema informativo a carico di Addetti di Sala.
	\item Preparazione delle pietanze di suddette ordinazioni a carico di Addetti alla cucina.
\end{itemize}

Le pagine a seguire si preoccuperanno di descrivere nel dettaglio le scelte progettuali-implementative al fine di realizzare quanto descritto.
 
