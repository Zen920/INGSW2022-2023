\chapter[Documento di]{Design del sistema}(midpoint-6)

\section{Architettura del sistema}
	\subsection{Tecnologie utilizzate}
	Dando assoluta priorità alle principali funzionalità richiesteci e agli interessi dei consumatori finali dell'applicazione, ossia i dipendenti dell'attività
	di ristorazione, abbiamo optato per le seguenti tecnologie, per ognuna delle quali verrà fornita una descrizione dettagliata e il corrispondente movente.

	\begin{itemize}
		\item Il framework Java di maggior successo \href{https://spring.io/}{Spring Boot} che facilita nello sviluppo di un servizio \textbf{REST API}%
		\footnote{Restful API: interfaccia di programmazione che consente l'interazione con servizi web Restful che comunicano tramite richieste HTTP.}%
		\footnote{Application Interface: un insieme di definizioni e protocolli atti alla creazione e all'integrazione di software applicativi.}.%
		\item La libreria \href{https://reactjs.org/}{React} per lo sviluppo di un front-end con un'interfaccia semplice ed efficace.
		\item \href{https://www.docker.com/}{Docker} per il totale isolamento dei suddetti servizi (web-server con nginx, back-end con Spring Boot, database),
		nonché permettere il deployment indipendentemente dall'architettura della macchina host scelta, tramite virtualizzazione%
		\footnote{Virtualizzazione: Fondamentalmente si riferisce alla possibilità di astrarre dalle componenti hardware, creando uno o più sistemi \textit{virtuali}, non fisici, al fine di eseguire software
		ideato per il tipo di sistema virtuale.}.
	\end{itemize}

	\begin{info}[Configurazione Docker]
		In quanto la configurazione dei container docker utilizzati non è banale segue, una sintesi di quali container siano configurati.
		
		\begin{itemize}
			\item \textbf{Front-end} che si espone su di una porta di accesso (\textit{configurabile}); tale contenitore è autorizzato al collegamento
			al Back-end tramite una rete virtuale interna, che rende inaccessibile il back-end da accessi non autorizzati.
			\item \textbf{Back-end} che risulta esposto soltanto \textit{internamente} (cioè, solo sulla macchina con docker in esecuzione);
			tale container condivide una rete virtuale on il Front-end e una rete virtuale con il container del Database.
			\item \textbf{Database} anch'esso esposto soltanto \textit{internamente}; si interfaccia al back-end tramite una rete virtuale.
		\end{itemize}
	\end{info}

	\begin{itemize}[resume]
		\item \href{https://www.nginx.com/}{Nginx} per la creazione del Web Server atto alla gestione delle richieste front-end, 
		che reindirizza automaticamente e internamente le richieste al back-end; con tale scelta assicurata la \newline scalabilità
		\footnote{Scalabilità: capacità di un sistema informatico di aumentare o diminuirne la scala in funzione delle necessità e delle disponibilità.}.
		\item \href{https://www.postgresql.org/}{DBMS PostgreSQL}\footnote{DBMS: acronimo di Database Management System, ossia un sistema software progetto per la creazione, la manipolazione e l'interrogazione efficiente di database.}, 
		sistema di gestione di base di dati relazionale, open source, che presenta API dedicate in molteplici linguaggi di programmazione.
		% TODO:Aggiungi nome completo
		\item \href{https://cloud.google.com}{Google Cloud} per il deployment su macchina virtuale remota.
	\end{itemize}

	\newpage
	\subsection{Scelta dei linguaggi}
	Nel back-end, la scelta è ricaduta su \textbf{Java (JDK17)} che, tramite le apposite tecnologie sopracitate (cui framework Spring Boot e Gradle), ne ha facilitato
	e velocizzato lo sviluppo, soprattutto per quanto concerne l'autorizzazione e il \textbf{request-response} pattern\footnotemark, caratteristico delle \textbf{Rest API}.

	\footnotetext{Il request-response è una metodologia di comunicazione che prevede lo scambio di messaggi tra un computer che effettua una richiesta (request)
	e un altro che vi risponde con delle informazioni inerenti (response). Tale pattern è caratteristico dei servizi web HTTP.}

	Per il front-end si è optato sin da subito per tecnologie multipiattaforma affinché i fruitori di suddetta applicazione non risultino obbligati all'uso
	di certi determinati dispositivi; la scelta è dunque ricaduta sull'utilizzo dei linguaggi del web, in particolare \textbf{ReactJS (Javascript)}, \textbf{HTML}, \textbf{CSS}.

	La documentazione, questo documento, è interamente scritta in \textbf{LaTeX}, precisamente la scelta è ricaduta sull'engine \textbf{LuaLatex}, che estende
	ulteriormente il linguaggio, integrandolo a vero e proprio linguaggio di programmazione (\href{https://www.lua.org/}{Lua}), permettendo un ampio range di modifiche.

	La maggior parte dei diagrammi UML qui presenti sono stati descritti \textit{manualmente} tramite l'utilizzo di appositi librerie \textbf{\LaTeX}
	quali \href{https://perso.ensta-paris.fr/~kielbasi/tikzuml/index.php}{tikz-uml} e strumenti esterni \href{https://plantuml.com/}{plantuml}.

	Infine si è fatto ampio uso di \textbf{Python} scripting per procedure automatizzate di testing di autenticazione, nonché di
	altre operazioni dal back-end, e di \textbf{Bash} scripting per la configurazione, l'abilitazione e l'installazione on-the-fly dei sopracitati contenitori 
	docker richiesti.

	\subsection{Strumenti utilizzati}
	Nella scrittura del codice sorgente la scelta è ricaduta su \textit{svariati} strumenti di sviluppo, in particolare l'utilizzo di diversi
	IDE\footnotemark{IDE: acronimo di Integrated Development Environment, in italiano \textbf{Ambienti di Sviluppo}, indica una particolare suite di software
	che aumenta la produttività del programmatore mettendo a disposizione strumenti di profiling, testing e modifica, semplificandone il lavoro.}
	e editor di testo quali \textbf{IntelliJ Idea} e \textbf{Neovim}\footnote{Sebbene non sia \textit{un vero e proprio IDE}, con la giusta configurazione può essere utilizzato come tale}.
	
	Inoltre, lo sviluppo è stato caratterizzato dall'intenso utilizzo delle seguenti
	\begin{itemize}
		\item \href{https://gradle.org/}{Gradle} per l'automazione del processo di \textbf{build}, \textbf{packaging}, \textbf{deployment}.
		\item \href{https://www.postman.com/}{Postman} per la costruzione, dunque il testing di request http da inoltrare al back-end.
		\item \href{https://nodejs.org/en/}{NodeJS} per costruire rapidamente il front-end indipendentemente dall'esistenza del back-end per ottimizzarne i tempi di sviluppo.
	\end{itemize}

	\section{Macrofunzionalita'}
	Di seguito esplicitato, per ogni macro funzionalità, le specifiche di implementazione.


	\subsection{Registrazione e Autenticazione}
	Per la gestione di autenticazione di dipendenti in Ratatouille23 si è fatto uso di \textbf{Bearer Token}, conformi allo standard \textbf{OAuth2}\footnotemark,
	che permettono la gestione di sessioni di autenticazione.

	\footnotetext{OAuth2: acronimo di Open Authorization, uno standard di permanenza all'autenticazione di accesso alle risorse.}

	I dipendenti, registrati da un amministratore, saranno dunque salvati sul DBMS PostgreSQL nell'apposita tabella, che presenta vincoli comuni ad un sistema
	di autenticazione (lunghezza minima della password, nomi utenti univoci, \dots) e, alla corretta autenticazione tramite apposito form sul front-end, potranno
	dunque accedere alle funzionalità presenti. La sessione sarà terminata non appena il dipendente avrà scelto di effettuarne il logout\footnotemark.

	\footnotetext{Il logout è la procedura di uscita da un sistema informatico; in questo specifico caso corrisponde alla revoca della validità dell'authorization token.}

	Il processo di autenticazione è componente comune a tutti i dipendenti dell'attività, amministratori e non, i quali sono differenziati durante la loro creazione (e cioè all'inserimento
	nel sistema): per una migliore manutenibilità del codice si è infatti preferito accorpare le due funzionalità, soprattutto considerando che la \textbf{Security through obscurity}%
	\footnote{La sicurezza tramite segretezza è una metodologia che prevede di mantenere la sicurezza di un sistema tenendone \textit{segreto} il funzionamento; è spesso pratica scoraggiata e non raccomandata.}
	non corrisponde esattamente ad una soluzione adeguata e, inoltre, i campi richiesti alla procedura di registrazione e di autenticazione sono solo il minimo indispensabile\footnote{In quanto
	la registrazione è compito dell'amministratore, non è richiesto alcun tipo di controllo antispam o di riconferma dell'account tramite email o numero di telefono, ad esempio.}.
	
	\subsection{Statistiche dipendente}
	Durante l'utilizzo dell'applicazione da parte dei dipendenti categorizzati come \textbf{Addetti alla cucina}, saranno
	raccolte, nel database, alcune statistiche che espliciteranno, all'amministratore che ne fa richiesta, il rendimento
	del lavoratore preso in considerazione.

	Per la raccolta dei dati ci si è totalmente affidati al DBMS PostgreSQL mentre per la visualizzazione si è fatto ampiamente uso
	della libreria ReactJS \href{https://nivo.rocks}{Nivo}.

	\newpage
	\subsection{Amministrazione del Menu'}
	Il salvataggio delle pietanze nel menù è affidato al DBMS PostgreSQL, per la visualizzazione delle pietanze nella composizione dell'ordine si è fatto uso
	di componenti nativi\footnotemark ReactJS che permettano velocemente agli \textbf{Addetti alla sala} di adempiere al loro lavoro.
	
	\footnotetext{In questo caso per componente nativo s'intende una qualche implementazione built-in nel framework React.}

	Inoltre, tramite l'API Open Source\footnote{Open Source: spesso abbreviato in OS, indica un qualsiasi prodotto software 
	in cui ci è data la possibilità di visionare liberamente documentazione e codice sorgente del prodotto stesso.} OpenFoodFacts, sarà data
	la possibilità di autocompletare\footnote{Una feature che prevede \textit{tramite un algoritmo} le intenzioni dell'utente, nel nostro caso specifico,
	digitando alcune lettere e selezionando un alimento, verranno aggiunti alcune informazioni su di esso in automatico.} gli inserimenti, semplificandone il processo.

	\subsection{Gestione delle ordinazioni}
	I dipendenti della categoria \textit{Addetti alla Sala} saranno incaricati di aggiungere, nel sistema, le ordinazioni attraverso
	il software front-end Ratatouille23.

	Il sistema terrà conto, tramite il DBMS PostgreSQL e il Backend Spring Boot, quali ordini risultino incompleti, in preparazione o completati.

	\section{Design Pattern}
	\subsection{Back-end}[Architettura implementativa]
	Il back-end trae ispirazione dal modello \textbf{Microservices Architecture} che, in sintesi, permette ad un'applicazione
	completa di essere segmentata in tante parti indipendenti, ognuna caratterizzata da delle responsabilità: ogni componente
	del sistema è quindi mantenibile, scalabile nonché testabile e riutilizzabile come standalone\footnote{Standalone: capace di funzionare in totale autonomia dalle altre componenti.} qual'ora ce ne fosse la necessità.

	\begin{figure}[h]
		\begin{minipage}[t]{.25\textwidth}\vspace{0pt}
			\begin{tikzpicture}[remember picture]
				% Icon
				\node[inner sep=0pt] (root) at (0,0)
					{\color{\ddchaptercolor}\Materialaccounttree};
				\node[inner sep=0pt, anchor=west] (root-lbl) at ([xshift=3mm] root.east)
					{\small\texttt{backend/src/main}};

				% Subfolders
				\foreach \package[remember=\package as \lastp (initially root)] in {database,openfoodfacts,ratatouille23,utility}{
					\node[inner sep=0pt, anchor=north west] (\package) 
						at ([yshift=-2mm] \lastp.south west -| root-lbl.south west) {\color{\ddchaptercolor}\Materialfolder};
					\node[inner sep=0pt, anchor=west] (\package-lbl) at ([xshift=3mm] \package)
						{\small\texttt{\package}};
					\path[draw=\ddchaptercolor, line width=0.25pt] 
						(root.south) 
						-- (root.south |- \package.west)
						-- (\package.west);
				}
			\end{tikzpicture}
		\end{minipage}
		\hfill
		\begin{minipage}[t]{0.70\textwidth}\vspace{0pt}
			Vi è una forte separazione delle componenti logiche-amministrative proprie (cioè, da noi sviluppate), racchiudendole
			nel loro specifico Java package\footnotemark, dalle
			componenti secondarie riguardanti il DBMS PostgreSQL e l'integrazione dell'API OpenFoodFacts. 
			Inoltre, per tutte le funzionalità non categorizzabili, si è preferito l'utilizzo di
			un package dedicato.
		\end{minipage}\hfill\null
	\end{figure}

	\footnotetext{Terminologia propria di Java che sta ad indicare un raggruppamento logico di classi, allo scopo di contestualizzarle.}

	\subsection{Implementazione del back-end}[Model-View-Controller]
	L'implementazione è avvenuta rispettando il Model-View-Controller pattern, tipico delle applicazioni
	Java Spring Boot che prevede la separazione delle componenti ai fini di semplificarne lo sviluppo.

	Nel dettaglio riportiamo, al momento della stesura di \textit{questo} documento, lo stato del progetto
	nonché la logica implementativa.

	\vfill\begin{figure}[!b]
		\centering\begin{tikzpicture}
			% Coordinate system
			\coordinate		(constraint-uleft) at (0, 0);
			\coordinate 	(constraint-dright) at (\textwidth, 0);

			\begin{pgfonlayer}{background}
				\fill[\ddchaptercolor!10!white]	(constraint-uleft) rectangle ([yshift=0cm] constraint-dright);
			\end{pgfonlayer}

			\begin{scope}[remember picture, shift=({constraint-uleft}), x={(constraint-uleft -| constraint-dright)}, y={(constraint-uleft |- constraint-dright)}, mcomp/.style={nosep, fill=ddchaptercolor!20!white, draw=ddchaptercolor!60!black, minimum width=4cm, minimum height=0.75cm}, unidir/.style={draw=ddchaptercolor!80!black, line width=1.25pt, >=triangle 45, ->}]
				\node[mcomp] 
					(model) at (0.5,0)																		{Model};
				\node[mcomp] 
					(view) at ([yshift=-3cm] 0.25,0 |- model)							{View};
				\node[mcomp] 
					(controller) at ([yshift=-3cm] 0.75,0 |- model)				{Controller};

				\path[unidir] (view.east)
					-- ([xshift=-0.75cm] view.east -| model.south)
					-- ([xshift=-0.75cm] model.south);

				\path[unidir] (controller.west)
					-- ([xshift=0.75cm] controller.west -| model.south)
					-- ([xshift=0.75cm] model.south);

				\path[unidir] (model.west)
					-- (model.west -| view.north)
					-- (view.north);

				\path[unidir] (model.east)
					-- (model.east -| controller.north)
					-- (controller.north);
			\end{scope}
		\end{tikzpicture}
		\caption{Rappresentazione grafica del pattern Model-View-Controller.}
	\end{figure}


	\newpage\begin{figure}[h]
		\begin{minipage}[t]{.25\textwidth}\vspace{0pt}
			\begin{tikzpicture}[remember picture]
				% Icon
				\node[inner sep=0pt] (root) at (0,0)
				{\color{\ddchaptercolor}\Materialaccounttree};
				\node[inner sep=0pt, anchor=west] (root-lbl) at ([xshift=3mm] root.east)
				{\small\texttt{b/s/main/ratatouille23}};

				% Subfolders
				\foreach \package/\ext/\icon[remember=\package as \lastp (initially root)] in {
					Configuration//\Materialfolder,
					Controllers//\Materialfolder,
					Entities//\Materialfolder,
					Repositories//\Materialfolder,
					Security//\Materialfolder,
					Service//\Materialfolder,
					Utilities//\Materialfolder,
					App/.java/\Brandsjava
				}{
					\node[inner sep=0pt, anchor=north west] (\package) 
					at ([yshift=-2mm] \lastp.south west -| root-lbl.south west) {\color{\ddchaptercolor}\icon};
					\node[inner sep=0pt, anchor=west] (\package-lbl) at ([xshift=3mm] \package)
					{\small\texttt{\package\ext}};
					\path[draw=\ddchaptercolor, line width=0.25pt] 
					(root.south) 
					-- (root.south |- \package.west)
					-- (\package.west);
				}

			\end{tikzpicture}
		\end{minipage}
		\hfill
		\begin{minipage}[t]{0.70\textwidth}\vspace{0pt}
			\def\sepline{\begingroup\color{\ddchaptercolor}\vspace{6pt}\hrule height 1pt depth 0pt width \textwidth\relax\vspace{6pt}\endgroup}
			\begin{description}[leftmargin=!,labelwidth=\widthof{\bfseries Configuration},font=\color{ddchaptercolor!70!black}]
				\item[Configuration] Package costituito dalle classi di configurazione riguardanti Java Spring Boot.
					\sepline
				\item[Controllers] Package costituito delle routes (rotte) 
					per le chiamate HTTP effettuate dal client (front-end) affinché sia effettuabile la 
					comunicazione tramite request-response.
					\sepline
				\item[Entities] Package per la gestione dei modelli di trasmissione dati necessari affinché i controllers
					possano comunicare con il client.
					\sepline
				\item[Security] Package della gestione del modello di autenticazione tramite \textbf{OAuth2} Token, quindi quali routes
					siano accessibili tramite le HTTP requests e con quali permessi.
					\sepline
				\item[Repositories] Package contenente la logica di messa a disposizione di Java Spring Boot
					e il DBMS PostgreSQL per il salvataggio (quindi l'inserimento), l'aggiornamento, la cancellazione
					e la selezione (query) dei dati.
					\sepline
				\item[Service] Package contenente la logica di gestione delle classi Controllers.
					\sepline
				\item[Utilities] Package contenente configurazioni e funzionalità aggiuntive non riguardanti alcuna classe nello specifico, come ad esempio Java Exception personalizzate, Data-Validators aggiuntivi e costruzione, quindi segnalazione, di richieste ill-formed.
					\sepline
				\item[App.java] Corrisponde all'entry point dell'applicazione, eseguirà l'istanza di
					Java Spring Boot come da noi configurata.
			\end{description}
		\end{minipage}\hfill\null
	\end{figure}

	\newpage
	\subsection{Implementazione del Front-end}[Flux Pattern]
	Per l'applicazione front-end multipiattaforma, in quanto pattern tipico di ReactJS,
	l'implementazione è avvenuta rispettando il Flux pattern.

	In ReactJS è infatti preferibile \textit{passare} le informazioni da componente (padre) in componente (figlio):
	l'idea principale è che le informazioni viaggino in una sola direzione.

	\vspace{1cm}\begin{figure}[!b]
		\centering\begin{tikzpicture}
			% Coordinate system
			\coordinate		(constraint-uleft) at (0, 0);
			\coordinate 	(constraint-dright) at (\textwidth, 0);

			\begin{pgfonlayer}{background}
				\fill[\ddchaptercolor!10!white]	(constraint-uleft) rectangle ([yshift=0cm] constraint-dright);
			\end{pgfonlayer}

			\begin{scope}[remember picture, shift=({constraint-uleft}), x={(constraint-uleft -| constraint-dright)}, y={(constraint-uleft |- constraint-dright)}, mcomp/.style={nosep, fill=ddchaptercolor!20!white, draw=ddchaptercolor!60!black, minimum width=4cm, minimum height=0.75cm}, unidir/.style={draw=ddchaptercolor!80!black, line width=1.25pt, >=triangle 45, ->}]
				\node[mcomp] 
					(action) at (0.1,0)																		{Action};
				\node[mcomp] 
					(dispatcher) at (0.375,0)															{Dispatcher};
				\node[mcomp] 
					(store) at (0.625,0)																	{Store};
				\node[mcomp] 
					(view) at (0.9,0)																			{View};
				\node[mcomp]
					(action-two) at ([yshift=2cm] store)						{Action};

				\path[unidir] (action.east) -- (dispatcher.west);
				\path[unidir] (dispatcher.east) -- (store.west);
				\path[unidir] (store.east) -- (view.west);
				\path[unidir, -] (view.north) -- (view.north |- action-two.east) -- (action-two.east);
				\path[unidir] (action-two.west) -- (dispatcher.north |- action-two.west) -- (dispatcher.north);
			\end{scope}
		\end{tikzpicture}
		\caption{Rappresentazione grafica del Flux pattern.}
	\end{figure}\vfill

	\begin{figure}[h]
		\begin{minipage}[t]{.25\textwidth}\vspace{0pt}
			\begin{tikzpicture}[remember picture]
				% Icon
				\node[inner sep=0pt] (root) at (0,0)
				{\color{\ddchaptercolor}\Materialaccounttree};
				\node[inner sep=0pt, anchor=west] (root-lbl) at ([xshift=3mm] root.east)
				{\small\texttt{frontend/src}};

				% Subfolders
				\foreach \package/\ext/\icon[remember=\package as \lastp (initially root)] in {
					components//\Materialfolder,
					config//\Materialfolder,
					containers//\Materialfolder,
					hooks//\Materialfolder,
					images//\Materialfolder,
					routes//\Materialfolder,
					services//\Materialfolder,
					utilities//\Materialfolder,
					App/.css/\Brandscss,
					App/.js/\Brandsjavascript,
					i18n/.js/\Brandsjavascript,
					index/.js/\Brandsjavascript,
					index/.css/\Brandscss,
					logo/.svg/\Brandssvg
				}{
					\node[inner sep=0pt, anchor=north west] (\package) 
					at ([yshift=-2mm] \lastp.south west -| root-lbl.south west) {\color{\ddchaptercolor}\icon};
					\node[inner sep=0pt, anchor=west] (\package-lbl) at ([xshift=3mm] \package)
					{\small\texttt{\package\ext}};
					\path[draw=\ddchaptercolor, line width=0.25pt] 
					(root.south) 
					-- (root.south |- \package.west)
					-- (\package.west);
				}

			\end{tikzpicture}
		\end{minipage}
		\hfill
		\begin{minipage}[t]{0.70\textwidth}\vspace{0pt}
			\def\sepline{\begingroup\color{\ddchaptercolor}\vspace{6pt}\hrule height 1pt depth 0pt width \textwidth\relax\vspace{6pt}\endgroup}
			\begin{description}[leftmargin=!,labelwidth=\widthof{\bfseries components},font=\color{ddchaptercolor!70!black}]
				\item[components] Tale directory è costituita dalle componenti visive che formeranno, tramite programmazione
					per composizione\footnotemark tutte le pagine visualizzabili del front-end. 

					In particolare non ne decide la logica, ma soltanto come verranno impostati e posizionati (layout) tali componenti.
					\sepline
				\item[config] La directory comprende le configurazioni necessarie affinchè il front-end sia utilizzabile
					a seconda dell'ambiente (dev, prod).
					\sepline
				\item[containers] Nella directory sono presenti \textit{contenitori}, che aggregano componenti e logica.
					\sepline
				\item[hooks] La directory conterrà gli hooks dell'applicativo: un \textbf{hook}	(dall'Inglese uncino) modifica il comporamento di alcune delle funzionalità
				andando ad intercettarne il funzionamento originale, quindi aggiungendo chiamate a funzioni addizionali.
					\sepline
				\item[images] La directory contiene, per l'appunto, \textit{alcune} immagini che mirano a migliorare
				l'esperienza utente, fornendo al front-end un po' più di leggerezza.
					\sepline
				\item[routes] Le routes configurano la navigazione utente all'interno dell'applicazione ReactJS: a seconda
				del livello di autorizzazione ottenuto (e.g. non autorizzato, amministratore, addetto alla sala, \dots),
				l'utente avrà la possibilità di navigare in più o meno servizi (cioè, pagine) dell'applicativo front-end.
					\sepline
				\item[services] La directory si compone di tutta l'effettiva logica dell'applicazione ReactJS, tra cui
				le chiamate effettive alla REST API, nonchè l'autenticazione e la connessione al servizio.
					\sepline
				\item[utilities] La directory contiene principalmente script ri-utilizzabili in diversi ambienti 
				dell'applicativo; in particolare si è preferito centralizzare (cioè, raccogliere in files comuni)
				gli URL delle richieste alla REST API e alcuni messaggi.
					\sepline
				\item[App.css, Index.css] Fogli di stile con le proprietà inerenti all'entrypoint dell'applicazione.
				\item[App.js, Index.js] Configurazione dell'applicazione frontend ReactJS; costituiscono l'entrypoint logico dell'applicazione.
					\sepline
				\item[i18n.js] Script di configurazione della localizzazione a carico di \href{https://www.i18next.com/}{i18next}.
					\sepline
				\item[logo.svg] Icona visualizzata nel tab del browser per l'applicazione (favicon).
			\end{description}
		\end{minipage}\hfill\null
	\end{figure}
	
	\footnotetext{Composition: La programmazione per composizione è una metodologia di programmazione il cui principio
	assomiglia al polimorfismo ma, piuttosto che costruire in maniera incrementale, si \textbf{compone} dei diversi
	componenti, garantendo un certo livello di riutilizzabilità del codice e nel complesso anche una migliore \textit{pulizia del codice}}


	\section{Activity Diagrams}
	\subsection{Registrazione dipendente}[Inerente al requisito funzionale \ref{reqsf:register}]
	\svg{./src/diag/act/fr01}

	\newpage\subsection{Disattivazione dei dipendenti}[Inerente al requisito funzionale \ref{reqsf:unregister}]
	\svg{./src/diag/act/fr02}

	\newpage\subsection{Visualizzazione del rendimento}[Inerente al requisito funzionale \ref{reqsf:stats}]
	\svg{./src/diag/act/fr03}

	\newpage\subsection{Inserimento pietanze nel menù culinario}[Inerente al requisito funzionale \ref{reqsf:addmenu}]
	\svg{./src/diag/act/fr04}

	\newpage\subsection{Rimozione pietanze dal menù culinario}[Inerente al requisito funzionale \ref{reqsf:delmenu}]
	\svg{./src/diag/act/fr05}

	\newpage\subsection{Aggiornamento del menù culinario}[Inerente al requisito funzionale \ref{reqsf:chgmenu}]
	\svg{./src/diag/act/fr06}

	\newpage\subsection{Inserimento di nuove categorie alimentari}[Inerente al requisito funzionale \ref{reqsf:addcat}]
	\svg{./src/diag/act/fr07}

	\newpage\subsection{Cancellazione delle categorie alimentari esistenti}[Inerente al requisito funzionale \ref{reqsf:delcat}]
	\svg{./src/diag/act/fr08}

	\newpage\subsection{Aggiornamento delle categorie alimentari}[Inerente al requisito funzionale \ref{reqsf:chgcat}]
	\svg{./src/diag/act/fr09}

	\newpage\subsection{Aggiunta dei tavoli}[Inerente al requisito funzionale \ref{reqsf:addtbl}]
	\svg{./src/diag/act/fr10}

	\newpage\subsection{Rimozione dei tavoli}[Inerente al requisito funzionale \ref{reqsf:deltbl}]
	\svg{./src/diag/act/fr11}

	\newpage\subsection{Rimozione dei Inserimento avvisi nel sistema}[Inerente al requisito funzionale \ref{reqsf:addntf}]
	\svg{./src/diag/act/fr12}

	\newpage\subsection{Autenticazione}[Inerente al requisito funzionale \ref{reqsf:authlogin}]
	\svg{./src/diag/act/fr13}

	\newpage\subsection{Aggiornamento delle credenziali d'accesso alla prima autenticazione}[Inerente al requisito funzionale \ref{reqsf:authupdate}]
	\svg{./src/diag/act/fr14}

	\newpage\subsection{Nascondere avvisi}[Inerente al requisito funzionale \ref{reqsf:hidentf}]
	\svg{./src/diag/act/fr15}

	\newpage\subsection{Registrare ordini}[Inerente al requisito funzionale \ref{reqsf:addorder}]
	\svg{./src/diag/act/fr16}

	\newpage\subsection{Completa transazione}[Inerente al requisito funzionale \ref{reqsf:termorder}]
	\svg{./src/diag/act/fr17}

	\newpage\subsection{Accetta ordine}[Inerente al requisito funzionale \ref{reqsf:acptorder}]
	\svg{./src/diag/act/fr18}

	\newpage\subsection{Completa ordine}[Inerente al requisito funzionale \ref{reqsf:cmporder}]
	\svg{./src/diag/act/fr19}

	\newpage\subsection{Rilascia ordine}[Inerente al requisito funzionale \ref{reqsf:abrtorder}]
	\svg{./src/diag/act/fr20}

	\section{Design Sequence Diagrams}
	\subsection{Inserimento pietanze nel menu' culinario}[Inerente al requisito funzionale \ref{reqsf:register}]
	\svg{./src/diag/seq-design/fr04}

	\subsection{Inserimento dipendenti nel sistema}[Inerente al requisito funzionale \ref{reqsf:addmenu}]
	\svg{./src/diag/seq-design/fr01}

	\subsection{Registrazione nuovi ordini}[Inerente al requisito funzionale \ref{reqsf:addorder}]
	\svg{./src/diag/seq-design/fr16}

	\subsection{Completamento e Rilasciamento degli ordini}[Inerente ai requisiti funzionale \ref{reqsf:cmporder}, \ref{reqsf:abrtorder}]
	\svg{./src/diag/seq-design/fr19-20}

	\section{State Design Diagrams}
	\subsection{Registrazione dipendente}[Inerente al requisito funzionale \ref{reqsf:register}]
	\svg[0.7\measurepage]{./src/diag/st-design/fr01}

	\subsection{Autenticazione}[Inerente al requisito funzionale \ref{reqsf:authlogin}]
	\svg[0.7\measurepage]{./src/diag/st-design/fr13}

	\subsection{Registrazione nuovi ordini e completamento transazioni}[Inerente ai requisiti funzionali \ref{reqsf:addorder},\ref{reqsf:termorder}]
	\svg[0.7\measurepage]{./src/diag/st-design/fr16-17}

	\subsection{Completamento e Rilascio ordini}[Inerente ai requisiti funzionali \ref{reqsf:cmporder}, \ref{reqsf:abrtorder}]
	\svg[0.7\measurepage]{./src/diag/st-design/fr19-20}*
